\documentclass{article}

\usepackage{amsmath}
\usepackage{epigraph}
\usepackage{csquotes}

\usepackage{hyperref}
\hypersetup{pageanchor=false}



\title{Monoids - best API ever!}
\date{2010-09-30}
\author{Marek Dudek}

\begin{document}

\pagenumbering{gobble}
\maketitle

\newpage
\tableofcontents

\newpage
\pagenumbering{arabic}

\section{Motivation}

    \epigraph{The monoid is a humble algebraic structure, at first glance even downright boring. However, there’s much more to monoids than meets the eye.}

    \paragraph{}
    Understanding monoids is crucial for building distributed software systems that actually work as intended. 
    Monoids help us in thinking about concurrency and parallelism and thus about distribution.

    \paragraph{}
    Using monoid is getting more for less!

    

\section{Definitions}

\subsection{Semigroup}

    \paragraph{}
    Semigroup is an algebraic structure consisting of a set together with an associative binary operation.

    \begin{align*}
        &(A, \diamond) && \text{set and operator} \\
        a \diamond a &= a && \text{binary operator} \\
        x \diamond (y \diamond z) &= (x \diamond y) \diamond z && \text{associativity law}
    \end{align*}

    \paragraph{}
    Symbol $\diamond$ is often pronounced \textit{append}.

\subsection{Monoid}

    \paragraph{}
    Monoid is a semigroup with identity element.

    \begin{align*}
        \emptyset &\in A \\
        \emptyset \diamond a &= a && \text{left  identity} \\
        a \diamond \emptyset &= a && \text{right identity} 
    \end{align*}

    \paragraph{}
    Symbol $\emptyset$ is often pronounced \textit{empty} or \textit{neutral} element.

\subsection{Concatenation}

    \paragraph{}
    Once we have $\diamond$ operator for two elements we can define concatenation function for multiple elements.

    \begin{align*}
        concat([a_1, a_2, \dots, a_n]) &= a_1 \diamond (a_2 \diamond (\dots (a_{n-1} \diamond a_n))) \\
        concat(l) &= fold_{right}(\diamond, \emptyset, l)
    \end{align*}

    \paragraph{}
    For monoids concatenation of empty list is by definition equal to $\emptyset$.
    Semigroups don't have $\emptyset$ so this function is only defined for a non-empty list.

    \paragraph{}
    Actually most of the time this concatenation of elements of $A$ is what we are really after. We often operate on multiple values of the same type. We just want to boil it down to operating on only two values and extrapolate it to operating on any number of them. This simplifies matter a lot.

    \paragraph{}
    Appilcations of $concat$ can be easily parallelized, recomputed incrementally or cached.


\section{Simple examples}

    \subsection{Semigroup examples}

    \subsubsection{Minimum and maximum of numbers}

    \begin{align*}
        min(min(a, b), c) &= min(a, min(b, c)) \\
        max(max(a, b), c) &= max(a, max(b, c))
    \end{align*}

    We were talking about operators and we are used to $min$ and $max$ being functions. 
    This is only syntax, semantically binary operators and functions of two arguments returning the same type are equivalent:

    \begin{align*}
        min(min(a, b), c) &= (a \min b) \min c \\ 
        max(max(a, b), c) &= (a \max b) \max c 
    \end{align*}

    Minimum and maximum don't have neutral element. 
    We could say that $-\infty$ and $+\infty$ play this role. 
    But they aren't real numbers, they don't belong to a set on which we defined our operator.

    \subsection{Monoid examples}

    \subsubsection{Addition and multiplication of numbers}

    \begin{align*}
        (a + b) + c &= a + (b + c) \\
        (a * b) * c &= a * (b * c)
    \end{align*}
    \begin{align*}
        0 + a = a \\
        1 * a = a  
    \end{align*}

    \subsubsection{Logical disjunction and conjunction}

    \begin{align*}
        (a \vee   b) \vee   c &= a \vee   (b \vee   c) \\
        (a \wedge b) \wedge c &= a \wedge (b \wedge c)
    \end{align*}
    \begin{align*}
        false \vee   a = a \\
        true  \wedge a = a 
    \end{align*}

    \subsubsection{Union and intersection of sets}

    \begin{align*}
        (a \cup b) \cup c &= a \cup (b \cup c) \\
        (a \cap b) \cap c &= a \cap (b \cap c)
    \end{align*}
    \begin{align*}
        \emptyset \cup a = a \\
        \Omega    \cap a = a 
    \end{align*}

    \subsubsection{Appending lists}

    \begin{align*}
        (l_1 \otimes l_2) \otimes l_3 &= l_1 \otimes (l_2 \otimes l_3) \\
        [] \otimes l = l \\
    \end{align*}

    \subsubsection{Appending strings}

    \begin{align*}
        (s_1 \otimes s_2) \otimes s_3 &= s_1 \otimes (s_2 \otimes s_3) \\
        "" \otimes s = s \\
    \end{align*}

    \paragraph{}
    This works because we can treat string as list of characters.

    \section{More fun with semigroups and monoids}
    \subsection{Dual}
    \subsection{Turning semigroup into monoid}

    \paragraph{} 
    Trivial lifted monoid on values that have semigroup as wrapper for option type.

    \subsection{Other wrapper option types}

    \paragraph{}
    First and Last wrappers. Don't require that type of wrapped type have semigroup because they don't perform any operations on them.


    \section{More semigroups (that are not monoids)}

    \paragraph{}
    Adding positive integers is a semigroup, since $0$ is not included in the set.

    \paragraph{}
    Appending of non-empty lists is a semigroup, since empty list is not included in the set.

    \section{More monoids}

    \subsection{Logic}

    Exclusive disjunction and exclusive conjunction are also monoids.

    \subsection{Orderig}

    Ordering type has a monoid that is useful for comparisons by multiple criteria.

    \newpage

    \section{Function is a monoid!}

    \subsection{}
    For a given monoid M set of all functions from a given set to M is also a monoid. 
    Identity is a constant function mapping any value to the identity of M. Operation is defined pointwise.

    \subsection{}
    Let S be a set. The set of all functions $S -> S$ forms a monoid under function composition. Identity is just the identity function.

    \subsection{Functions to monoid}

    \subsection{Endomorphisms}

    \section{Applications of semigroups and monoids}

    \subsection{Foldable}

    \textit{Folding} is combining a function and a data structure, typically a list of elements. Fold combines elements of this data structure using given function in \textit{some} systematic way. It processes a data structure in \textit{some} order and builds a return value. This is as opposed to \textit{unfolding} which takes a starting value and applies a function to generate a data structure. In case of \textit{associative binary operation} to rough approximation you can think as replacing commas in the list with $\diamond$ operation. If $\diamond$ is \textit{associative binary operation} parentheses don't matter. In case of a function of two arguments of different type or a binary operator that is not associative But how exactly result of processing is calculated differs by operation details. One of these details is the order another is lazyness.

    Last part is definitely wrong.

    dimensions of difference:
        associative vs not
        or even not binary operator but a function a , b -> b (or in reverse)
        left vs right
        reactions to infinite structure, works lazily or halts?
        eagerness of applying function
        what is reason and what is criterion of choice?


        in right fold initial value used last, in left fold the first
        Initial value is put right in right, left in left.
        First function application is on the right is right, to last elements.
        First function application is on the left is left, to first elements.
        Left successively applies function from left (beginning) to right (end), accumulating result.
        Right goes to the end and applies function backtracking from right (end) to left (start).
        Right cannot work with infinite list, left can. Right would go in recursive calls indefinately without reaching funcion application. Left would accumulate value from the beginning, still in recursive calls but it wouldn't blow the stack (or however it was implemented).
        no, above is again wrong.

        magmas - foldrn - function assymetrical in its types
            

    \subsection{Monoid subclasses}

    \subsection{Monoid actions}

    \subsection{Monoid for applicatives}

    \subsection{Monoid homomorphisms}

    \subsection{Writer monad}

    \subsection{Finger trees}

    \subsection{Recreational math}

    Water monoid. \url{https://chris-martin.org/2017/water-monoids}

    

    \subsection{Options and settings}

    In Cabal, something like Maven for Haskell: 

    \blockquote{Package databases are monoids. Configuration files are monoids. Command line flags and sets of command line flags are monoids. Package build information is a monoid}

    In XMonad, windown manager implemented in Haskell:

    \blockquote{xmonad configuration hooks are monoidal}

    \subsection{Features}

    In GenVoca, a compositional paradigm for defining programs of product lines:

    \blockquote{model is a set of features with a composition operation}

    \section{Concurrency}

    Distinction between data and codata and the role of monoids in it.
    Lambda architecture vs Kudu vs abstract algebra.

    \newpage

    \section{Useful links}

    \begin{itemize}
        \item \url{https://en.wikibooks.org/wiki/Haskell/Monoids}
        \item \url{http://blog.ploeh.dk/2018/03/12/composite-as-a-monoid/}
        \item \url{https://bartoszmilewski.com/2017/02/09/monoids-on-steroids/}
    \end{itemize}




\end{document}
