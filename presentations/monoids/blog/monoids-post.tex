\documentclass{article}

\title{Monoids - best API ever!}
\date{2010-09-30}
\author{Marek Dudek}

\begin{document}

\pagenumbering{gobble}
\maketitle

\newpage
\tableofcontents

\newpage
\pagenumbering{arabic}

\section{Definitions}

\subsection{Semigroup}

    \paragraph{}
    Semigroup is an algebraic structure consisting of a set together with an associative binary operation.

    \begin{equation}
        (A, \diamond)
    \end{equation}

    \begin{equation}
        a \diamond a = a
    \end{equation}

    \begin{equation}
        x \diamond (y \diamond z) = (x \diamond y) \diamond z
    \end{equation}

    \paragraph{}
    Symbol $\diamond$ is often pronounced \textit{append}.

\subsection{Monoid}

    \paragraph{}
    Monoid is a semigroup with identity element.

    \begin{equation}
        a \diamond \emptyset = a
    \end{equation}

    \begin{equation}
        \emptyset \diamond a = a
    \end{equation}

    \paragraph{}
    Symbol $\emptyset$ is often pronounced \textit{empty}.

\subsection{Concatenation}

    \paragraph{}
    Once we have $\diamond$ operator for two elements we can define concatenation function for multiple elements.

    \begin{equation}
        concat([a_1, a_2, ..., a_n]) = a_1 \diamond a_2 \diamond ... \diamond a_n
    \end{equation}

    \begin{equation}
        concat(as) = fold_{right}(\diamond, \emptyset, as)
    \end{equation}

    \paragraph{}
    For monoids concatenation of empty list is by definition equal to $\emptyset$.
    Semigroups don't have $\emptyset$ so this function is only defined for a non-empty list. 

\section{Simple examples}


\newpage
\section{Random notes}
\subsection{List of Semigroups and Monoids}

\begin{itemize}
    \item addition of numbers
    \item multiplication of numbers
    \item appending lists
    \item min and max of numbers
\end{itemize}


\end{document}
