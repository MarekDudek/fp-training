\documentclass{article}

\usepackage{amsmath}

\title{Monoids - best API ever!}
\date{2010-09-30}
\author{Marek Dudek}

\begin{document}

\pagenumbering{gobble}
\maketitle

\newpage
\tableofcontents

\newpage
\pagenumbering{arabic}

\section{Motivation}

    \paragraph{}
    Understanding these concepts is crucial for building distributed software systems that actually work as intended.

\section{Definitions}

\subsection{Semigroup}

    \paragraph{}
    Semigroup is an algebraic structure consisting of a set together with an associative binary operation.

    \begin{align*}
        &(A, \diamond) && \text{set and operator} \\
        a \diamond a &= a && \text{binary operator} \\
        x \diamond (y \diamond z) &= (x \diamond y) \diamond z && \text{associativity law}
    \end{align*}

    \paragraph{}
    Symbol $\diamond$ is often pronounced \textit{append}.

\subsection{Monoid}

    \paragraph{}
    Monoid is a semigroup with identity element.

    \begin{align*}
        \emptyset &\in A \\
        \emptyset \diamond a &= a && \text{left  identity} \\
        a \diamond \emptyset &= a && \text{right identity} 
    \end{align*}

    \paragraph{}
    Symbol $\emptyset$ is often pronounced \textit{empty} or \textit{neutral} element.

\subsection{Concatenation}

    \paragraph{}
    Once we have $\diamond$ operator for two elements we can define concatenation function for multiple elements.

    \begin{align*}
        concat([a_1, a_2, \dots, a_n]) &= a_1 \diamond a_2 \diamond \dots \diamond a_n \\
        concat(l) &= fold_{right}(\diamond, \emptyset, l)
    \end{align*}

    \paragraph{}
    For monoids concatenation of empty list is by definition equal to $\emptyset$.
    Semigroups don't have $\emptyset$ so this function is only defined for a non-empty list. 

\section{Simple examples}

    \subsection{Semigroup examples}

    \subsubsection{Minimum and maximum of numbers}

    \begin{align*}
        min(min(a, b), c) &= min(a, min(b, c)) \\
        max(max(a, b), c) &= max(a, max(b, c))
    \end{align*}

    We were talking about operators and we are used to $min$ and $max$ being functions. 
    This is only syntax, semantically binary operators and functions of two arguments returning the same type are equivalent:

    \begin{align*}
        min(min(a, b), c) &= (a \min b) \min c \\ 
        max(max(a, b), c) &= (a \max b) \max c 
    \end{align*}

    Minimum and maximum don't have neutral element. 
    We could say that $-\infty$ and $+\infty$ play this role. 
    But they aren't real numbers, they don't belong to a set on which we defined our operator.

    \subsection{Monoid examples}

    \subsubsection{Addition and multiplication of numbers}

    \begin{align*}
        (a + b) + c &= a + (b + c) \\
        (a * b) * c &= a * (b * c)
    \end{align*}
    \begin{align*}
        0 + a = a \\
        1 * a = a  
    \end{align*}

    \subsubsection{Logical disjunction and conjunction}

    \begin{align*}
        (a \vee   b) \vee   c &= a \vee   (b \vee   c) \\
        (a \wedge b) \wedge c &= a \wedge (b \wedge c)
    \end{align*}
    \begin{align*}
        false \vee   a = a \\
        true  \wedge a = a 
    \end{align*}

    \subsubsection{Union and intersection of sets}

    \begin{align*}
        (a \cup b) \cup c &= a \cup (b \cup c) \\
        (a \cap b) \cap c &= a \cap (b \cap c)
    \end{align*}
    \begin{align*}
        \emptyset \cup a = a \\
        \Omega    \cap a = a 
    \end{align*}


    \subsubsection{Addition and multiplication of matrices}

    \paragraph{}
    By extension, addition and multiplication of matrices are also monoids
    with all-$0$ matrix and identity matrices as corresponding neutral elements.
    In this context we assume that addition is defined separately for matrices of the same dimentions 
    and multiplication is defined separately for square matrices of the same size.

    \subsubsection{Appending lists}

\end{document}
